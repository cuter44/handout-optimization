\documentclass{article}

\usepackage[a5paper,margin=1.25cm]{geometry}
%\setlength{\parindent}{0em}
%\setlength{\parskip}{0.8em}

\usepackage{xeCJK}
\setCJKmainfont{Noto Sans CJK SC}
\setCJKmonofont{Noto Sans Mono CJK SC}

\usepackage[hyphens]{url}
\usepackage[bookmarks=true]{hyperref}
\usepackage{spverbatim}

\usepackage{amsthm}
\theoremstyle{definition}
\newtheorem{definition}{定义}[section]
\theoremstyle{theorem}
\newtheorem{theorem}{定理}[section]

\usepackage{amssymb}
\newcommand{\Reals}{\mathbb{R}}
\usepackage{amsmath}

\title{最优化方法讲义}
\author{
  魏福义\ 著
  \and
  吴嘉林\ 编
}

\begin{document}

%\maketitle

\section{运筹学, 基本概念和理论基础}
\subsection{运筹学思想}

% 除去运筹学部分

\begin{definition}[优化模型的一般形式] 优化问题及其模型可以表示为以下的一般形式:
\begin{align*}
\begin{cases}
  \text{opt.} \; & f(x_i, y_j, \xi_k) \\
  \text{s.t.} \; & g_h(x_i, y_j, \xi_k) \leq 0 \; \text{, 或等于或大于等于0}\\
                 & h = 1,2,\dots,m
\end{cases}
\end{align*}
其中: \\ 称 \( x_i \) 为决策变量, 可人为调控, 亦是求解的目标; \\ 称 \( y_i \) 为已知参数; 称 \( \xi_k \) 为随机因素; \\ 称 \( g_h \dots \) 为限制条件. \\ \( f \) 和 \( g_h \) 为一般或广义函数.
\end{definition}

% 除去建模举例部分

\clearpage

\subsection{基本概念和符号}

\begin{definition}[n元函数] 记为
\( f(x): \Reals^n \rightarrow \Reals \)
\end{definition}

\begin{definition}[线性函数] 记为 
\begin{align*}
f(x) &= c^\mathsf{T}x+b \\
     &= \sum c_i x_i + b
\end{align*}
\end{definition}

\begin{definition}[二次函数] 记为
\begin{align*}
f(x) &= \frac{1}{2} x^\mathsf{T}Qx + c^\mathsf{T}x + b \\
     &= \frac{1}{2} \sum_i \sum_j a_{ij} x_i x_j + \sum c_i x_i + b
\end{align*}
其中 \( Q \) 是二次型.
\end{definition}

\begin{definition}[向量值函数] 记为
\( F(x) :\Reals^n \rightarrow \Reals^m \)
\end{definition}

\begin{definition}[向量值线性函数] 记为
\( F(x) = A x + b \) , 其中 \( A \in \Reals^{m \times n} \), \( x \in \Reals^n \), \( F(x), d \in \Reals^m \). \\
亦可视为 \( F(x) = (f_1(x), f_2(x), ..., f_m(x))^\mathsf{T} \), 其中 \( f_i(x) = a_i^\mathsf{T} x + d_i \), 其中 \( a_i^\mathsf{T} \) 是 \( A \) 的第 \( i \) 列向量.
\end{definition}

\begin{theorem}
任意线性变换均可以表示为变换矩阵.
\end{theorem}

\begin{definition}[多元函数的梯度] 对于n元函数 \( f(x): \Reals^n \rightarrow \Reals \), 记其梯度为
\( \nabla f(x) = (\frac{\partial f}{\partial x_1}, \frac{\partial f}{\partial x_2}, \dots, \frac{\partial f}{\partial x_n})^\mathsf{T} \). \\
对于线性函数 \( f(x) = c^\mathsf{T} x + b \), 有 \( \nabla f(x) = c^\mathsf{T} \). \\
对于二次函数 \( f(x) = x^\mathsf{T} Qx + c^\mathsf{T} x + b \), 有 \( \nabla f(x) = Qx +  c^\mathsf{T} \).
\end{definition}

\begin{definition}[向量值函数的梯度, Jacobian 矩阵] 对于向量值函数 \( F(x) :\Reals^n \rightarrow \Reals^m \), 记其梯度为
\begin{align*}
\nabla f(x) = \begin{bmatrix}
  \frac{\partial f_1}{\partial x_1} & \cdots & \frac{\partial f_1}{\partial x_n} \\
  \vdots                            & \ddots & \vdots                            \\
  \frac{\partial f_m}{\partial x_1} & \cdots & \frac{\partial f_m}{\partial x_n} 
\end{bmatrix}
\end{align*}
对于向量值线性函数 \( f(x) = Ax + b \), 有 \( \nabla f(x) = A^\mathsf{T} \).
\end{definition}

\begin{definition}[多元函数的二阶导数, Hessian 矩阵] 对于n元函数 \( f(x): \Reals^n \rightarrow \Reals \), 记其二阶导数为:
\begin{align*}
\nabla^2 f(x) = \begin{bmatrix}
  \frac{\partial f}{\partial x_1^2}            & \frac{\partial f}{\partial x_2 \partial x_1} & \cdots & \frac{\partial f}{\partial x_n \partial x_1} \\
  \frac{\partial f}{\partial x_1 \partial x_2} & \frac{\partial f}{\partial x_2^2}            & \cdots & \frac{\partial f}{\partial x_n \partial x_2} \\
  \vdots                                       & \vdots                                       & \ddots & \vdots                                       \\
  \frac{\partial f}{\partial x_1 \partial x_n} & \frac{\partial f}{\partial x_2 \partial x_n} & \cdots & \frac{\partial f}{\partial x_n^2} 
\end{bmatrix}
\end{align*}
对于线性函数 \( f(x) = c^\mathsf{T} x + b \), 有 \( \nabla^2 f(x) = 0^{} \). \\
对于二次函数 \( f(x) = x^\mathsf{T} Qx + c^\mathsf{T} x + b \), 有 \( \nabla^2 f(x) = Q \).
\end{definition}

\begin{definition}[多元函数的泰勒展开] 设有n元函数 \( f(x): \Reals^n \rightarrow \Reals \), 在 \( x^* \) 的邻域内 k 阶可导, 则 f(x) 可在该邻域内展开. \\
%\begin{align*}
%f(x) = \sum _{n_{1}=0}^{\infty }\cdots \sum _{n_{d}=0}^{\infty }{\frac {\partial ^{n_{1}+\cdots +n_{d}}}{\partial x_{1}^{n_{1}}\cdots \partial x_{d}^{n_{d}}}}{\frac {f(a_{1},\cdots ,a_{d})}{n_{1}!\cdots n_{d}!}}(x_{1}-a_{1})^{n_{1}}\cdots (x_{d}-a_{d})^{n_{d}}
%\end{align*}
一阶泰勒展开式的形式为:
\begin{align*}
f(x) \:=\: f(x^*) \:+\: \nabla f^\mathsf{T}(x^*)(x-x^*) \:+\: o\left(\lVert x-x^*\rVert\right)
\end{align*}
二阶泰勒展开式的形式为:
\begin{align*}
f(x) \:=\: & f(x^*) \:+\: \nabla f^\mathsf{T}(x^*) (x-x^*) \\
       & +\: \frac{1}{2} (x-x^*)^\mathsf{T} \nabla^2 f(x^*) (x-x^*) \:+\: o\left({\lVert x-x^*\rVert^2}\right)
\end{align*}
\end{definition}

\begin{definition}[多元函数的中值公式] 设有上述之多元函数, 有 \\
(一阶形式)
\begin{align*}
                  & \exists \lambda \in (0, 1), x_\lambda = x^*+\lambda(x-x^*)\\
\text{s.t.} \; & f(x) = f(x^*) + \nabla f^\mathsf{T}(x_\lambda) (x-x^*)
\end{align*}
(二阶形式)
\begin{align*}
                  & \exists \mu \in (0, 1), x_\mu = x^*+\mu(x-x^*)\\
\text{s.t.} \; & f(x) = f(x^*) + \nabla f^\mathsf{T}(x^*) (x-x^*) + \frac{1}{2} (x-x^*)^\mathsf{T} \nabla^2 f(x_\mu) (x-x^*)
\end{align*}
\end{definition}

\clearpage

\subsection{最优化方法的基本概念}

\begin{definition}[数学规划模型的一般形式, 最优解, 最优值] 设有集合 \( S \subseteq \Reals^n \), \( f:S \rightarrow \Reals \), 记数学规划模型为
\begin{align*}
(f\;S) \begin{cases}
\text{min} \:  & f(x) \\
\text{s.t.} \: & x \in S
\end{cases}
\end{align*}
其中, 称 \( f(x) \) 为目标函数, \( S \) 为约束集合, a.k.a. 可行集. \\
设有元素 \( x^* \in S \), \( \forall x \in S \), s.t. \( f(x^*) \leq f(x)\), 称 \( x^* \) 为 \( (f\;S) \)的全局最优解(记 g.opt.), a.k.a. 最优解(记 opt.). 称函数值 \( f(x^*) \) 为 \( (f\;S) \) 的最优值. \\
设有元素 \( x^* \in S \), 邻域\( N(x^*) \), \( \forall x \in S \cap N(x^*) \), s.t. \( f(x^*) \leq f(x)\), 称 \( x^* \) 为 \( (f\;S) \)的局部最优解(记 l.opt.). \\
在上述定义中,当 \( x \neq x^* \) 时有严格不等式成立,则分别称 \( x^* \) 为 \( (f S) \) 的严格全局最优解或严格局部最优解。
\end{definition}

\end{document}
